\documentstyle[note,harvard]{article}
\def\id#1{\hbox{\it#1}}
\parindent=0pt
\parskip= .35\baselineskip plus .0833333\baselineskip 
                          minus .0833333\baselineskip
\pagestyle{empty}
\begin{document}

\section*{ PRESS: PRolog Equation Solving System}

\subsection*{AUTHORS:}
 Leon Sterling, Alan Bundy, Lawrence Byrd, Richard O'Keefe, and
	 Bernard Silver

\subsection*{What is PRESS? }

Humans are surprisingly good at solving equations! Faced with a plethora
of rules they usually home on the most useful one. They may
be unaware of the wealth of knowledge they bring to bear on this
problem. For example, could you explain why ``addding 4 to both sides''
makes sense in
\[ x - 4 = 0 \]
but not in
\[ x^2 +3x - 4 = 0? \]

PRESS was developed in order to try out ideas about
the kind of knowledge that mathematicians use to guide proofs. Rather
than trying rules of inference at random, they use a deeper analysis
to reduce the number of possible rules. As a result, the output which
PRESS produces is quite human-like. The test bed for PRESS was 172
examination questions taken from `A' level and Higher papers, of which
PRESS was successful in 85\% of them - enough to get a Grade `A'!

PRESS works by categorizing the rules of algebra according to their
effects. For example, ``Isolation Axioms'' work when there is only
one occurence of the unknown in the equation (e.g. 3 + 2x = 11)
and their effect is to isolate the unknown on one side of the equation.
The ``Collection'' Method works to reduce the number of occurences
of the unknown. ``Attraction'' does not reduce the number of occurences,
but brings them ``closer together'' making it more likely that they
can then be ``collected''. More information can be found in the PRESS
literature: a good starting point is given below.
\subsection*{How to use PRESS }
The top-level predicate is solve/1, solve/2, or solve/3:
\begin{verbatim} 
 solve(Expression)                e.g. solve(1-3*cos(x)^2=5*sin(x)).
 solve(Expression,Unknown)        e.g. solve(1-3*cos(x)^2=5*sin(x),x).
 solve(Expression,Unknown,Answer) e.g. solve(1-3*cos(x)^2=5*sin(x),x,X).
\end{verbatim}

For simultaneous equations it is simsolve, e.g.
\begin{verbatim}
simsolve(x+y=5 & 2*x+y^2=18). 
\end{verbatim}	


\subsection*{Associated literature:}
 A good introduction is Sterling et al.(1982):
		       ``Solving Symbolic Equations with PRESS'',
		       DAI Research Paper 171.

\subsubsection*{CURRENT STATUS:}
  PRESS is freely available for academic use, but is
  		 unsupported. It is available via anonymous ftp from
		 dream.dai.ed.ac.uk (192.41.111.169) in the directory
		 pub/press - filename press.tar.Z. \\
CONTACT: Raul Monroy (raulm@aisb)





\end{document}
