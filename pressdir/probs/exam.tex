\documentstyle[note]{article}
\begin{document}
\LARGE

\subsection*{$\mid\;\;?-$ example1.}
This problem comes from the London 1978 A level exam.

We are asked to find the value(s) of for which
\[	\log_2(x) + 4\log_x(2) = 5\]


Rewriting equation in terms of $log_2(x)$ gives 
\[\log_2(x) + 4\log_2(x)^{-1} = 5\]

Substituting $x_1$ for $\log_2(x)$ gives
\[ 4x_1^{-1} + x_1 = 5\]

Multiply through by $x_1$ to get 
\[x_1^2 + -5x_1 + 4 = 0\]

Using quadratic equation formula. Solutions are $x_1 = 4$ and $x_1 = 1$

Applying substitution $x_1 = \log_2(x)$ to: $x_1 = 4\wedge x_1 = 1$ gives: 
\[    \log_2(x) = 4 \wedge \log_2(x) = 1\]

Solving disjunct $\log_2(x) = 4$, gives
\[    x = 16\]

(by Isolation)

Solving disjunct $\log_2(x) = 1$, gives
\[    x = 2\]

(by Isolation)

\vspace*{.25in}
Answer is : $(X1 \wedge X2)$; where: 
\begin{eqnarray*}
X1 & = & x = 16\\
X2 & = &  x = 2
\end{eqnarray*}

\newpage
\subsection*{$\mid\;\; ?-$ example2}

This problem is from the A.E.B. A level exam of 1971.

We are required to find the value(s) of x such that
\[	\cos(x) + 2\cos(2x) + \cos(3x) = 0\]

Angles are in arithmetic progression 

Adding in pairs
\[2\cos(2x) + 2\cos(2x)\cos(x) = 0\]

\[(2 + 2\cos(x))\cos(2x) = 0\]

Solving factor $\cos(2x) = 0$. Letting $n_1$ denote an arbitrary integer
\[x = 45 +\frac{180n_1}{2}\]

     (by Isolation)


Solving factor $2 + 2\cos(x) = 0$. Letting $n_2$ denote an arbitrary 
integer:
\[x = 180 + 360n_2 \wedge x = -180 + 360n_2\]

   (by Isolation)

\vspace*{0.25in}
Answer is : $(X1 \wedge (X2 \wedge X3))$ where :
\begin{eqnarray*}
    X1 & = &  x = 45 + 90n_1\\
    X2 & = & x = 180 + 360n_2\\
    X3 & = & x = -180 + 360n_2
\end{eqnarray*}
\newpage
\subsection*{$\mid\;\; ?-$ example3}

This problem is from the A.E.B. 1971 A level paper.

The question asks for the value(s) of $x$ which satisfy
\[4^x - 2^{x+1} - 3 = 0\]

Tidying to 
\[4^x + (-1)2^{1 + x} = 3\]

Rewriting equation in terms of $2^x$ gives 
\[(2^x)^2 + (-1)(2^{1}2^x) = 3\]

Substituting $x_2$ for $2^x$ gives
\[-2x_2 + x_2^2 = 3\]

Using quadratic equation formula. Solutions are $x_2 = 3$ and $x_2 = -1$.

Applying substitution $x_2 = 2^x$ to: $x_2 = 3 \wedge x_2 = -1$ gives: 
\[2^x = 3 \wedge 2^x = -1\]

Solving disjunct $2^x = 3$ we have 
\[x = \log_2(3)\]

(by Isolation)

Solving disjunct $2^x = -1$. $2^x = -1$ has no real roots, $2^x$ must be 
greater than 0.

Answer is: $X1$, where :
\[X1 =  x = \log_2(3) = 1.584962500721156E+00\]


\newpage
\subsection*{$\mid\;\; ?-$ example4}

This question demonstrates the basic methods of PRESS.

The problem is to find the value(s) of $x$ that satisfy
\[\log_e(x+1) + \log_e(x-1) = 3\]

Tidying to 
\begin{eqnarray*}
\log_e(1+x) + \log_e(-1+x) & = & 3\\
\log_e((-1 + x)(1 + x)) & = & 3\\
\log(e, -1 + x^2) & = & 3
\end{eqnarray*}

Finally
\[x = (1 + e^3)^{1/2} \wedge x = (-1)(1 + e^3)^{1/2}\]

(by Isolation)

Answer is : $(X1 \wedge X2)$ where:
\begin{eqnarray*}
X1 & = & x = (1 + e^3)^{1/2}\\
X2 & = & x = (-1)(1 + e^3)^{1/2}
\end{eqnarray*}

\end{document}
    X1 =  x = 4.591899054115592E+00
    X2 =  x = -4.591899054115592E+00


Answer is : 
(X1 \wedge X2)
  where :
    X1 =  x = 4.591899054115592E+00
    X2 =  x = -4.591899054115592E+00


